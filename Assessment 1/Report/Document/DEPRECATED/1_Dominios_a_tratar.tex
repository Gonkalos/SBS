\section{Domínios a tratar}
\subsection{Objetivos}
Este trabalho foi desenvolvido no âmbito na unidade curricular "Sistemas Baseados em Similaridade", do perfil "Machine Learning: Fundamentos e Aplicações", do Mestrado em Engenharia Informática, disponibilizado pela Universidade do Minho.

De forma paralela foram explorados dois \textit{datasets}, um fornecido pelos docentes da unidade curricular e um por nós escolhido. O \textit{dataset} fornecido visa registos de incidentes rodoviários na cidade de Braga, contendo dados como a magnitude do atraso gerado pelos incidentes, as estradas afetadas ou o atraso em segundos provocado pelo acidente. O nosso objetivo concreto é prevermos qual o nível de incidentes de cada registo com a maior precisão possível, este parâmetro pode ter 5 valores diferentes, entre "None" e "Very High".

O segundo \textit{dataset} utilizado incide sobre partidas de League of Legends, mais concretamente sobre os primeiros 10 minutos da mesma. League of Legends é um jogo multi-player online, em que duas equipas de 5 jogadores se defrontam. No mapa onde a partida acontece existem diversos objetivos espalhados que dão vantagens á equipa que os conquistar. O objetivo a que nos propomos consiste em tentar prever qual será a equipa vencedora, baseando esta previsão nos primeiros 10 minutos de jogo.

\subsection{Proposta de resolução}
Para atingir ambos os objetivos propomos-nos a desenvolver modelos de machine learning, através de \textit{Random Forest}, fazendo uso de conhecimentos adquiridos em aula. Para desenvolver estes modelos é utilizado Knime, sendo desenvolvidos diferentes processos para preparar os dados, optimizar os parâmetros do modelo e optimizar o mesmo.