\section{Introdução \& Contextualização}

Este trabalho foi desenvolvido no âmbito na unidade curricular "Sistemas Baseados em Similaridade", do perfil "Machine Learning: Fundamentos e Aplicações", do conjunto de cursos MIEI/MEI/MMC, disponibilizados pela Universidade do Minho.

De forma paralela foram explorados dois \textit{datasets}, um fornecido pelos docentes da unidade curricular e o outro escolhido por nós. O \textit{dataset} fornecido visa registos de incidentes rodoviários na cidade de Braga ao longo do ano de 2019, contendo dados como a magnitude do atraso gerado pelos incidentes, as estradas afectadas, o atraso em segundos provocado pelo acidente, entre muitos outros. O nosso objetivo é prever qual a magnitude de incidentes de cada registo com a maior precisão possível, este parâmetro pode assumir 5 valores distintos dentro do conjunto $C_1 = \{\mbox{\texttt{None}, \texttt{Low}, \texttt{Medium}, \texttt{High}, \texttt{Very\_High}}\}$.

O segundo \textit{dataset} utilizado incide sobre partidas do videojogo de \textit{League of Legends}, mais concretamente, sobre os primeiros 10 minutos das mesmas. \textit{League of Legends} é um jogo multi-player online, em que duas equipas de 5 jogadores se defrontam numa partida com uma duração média de 30 a 40 minutos. Com este estudo pretendemos perceber o quão forte é o efeito dos primeiros 10 minutos numa partida em garantir a vitória. 
