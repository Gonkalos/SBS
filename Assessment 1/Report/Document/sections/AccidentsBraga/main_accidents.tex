\section{Acidentes em Braga no ano de 2019}

% todo este conteudo, ate a proxima subsection, corresponde, na pratica,
% à componente de Business Understanding do CRISP-DM
Segundo a PORDATA\footnote{\url{https://www.pordata.pt/Portugal/Acidentes+de+via\%c3\%a7\%c3\%a3o+com+v\%c3\%adtimas++feridos+e+mortos+++Continente-326}}, no ano de 2018, em Portugal, foram registados $34235$ acidentes rodoviários com vítimas, totalizando $508$ vítimas mortais. Como tal, a motivação para o desenvolvimento de um trabalho deste tipo é imediata.

Ao conseguirmos compreender melhor os factores que causam acidentes, trabalhando em específico com os dados da zona de Braga, pretendemos ser capazes de generalizar e compreender quais atributos são os mais informativos no sentido de prever o número de acidentes num determinado dia.

Com um modelo capaz de efectivamente utilizarmos os atributos mais informativos, na prática seriamos capazes de desenvolver campanhas personalizadas para promover a segurança rodoviária, entre outras medidas cujo impacto possa ser directamente mensurável a partir do nosso modelo.

% STEP 2 IN CRISP DM METODOLY
\subsection{Compreensão dos Dados}
Os dados utilizados neste ponto foram cedidos pelos docentes da unidade curricular e versam incidentes rodoviários na cidade de Braga no ano de 2019.

Estes dados focam informações como por exemplo:
\begin{itemize}
    \item As estradas afetadas num determinado momento
    \item O atraso causado pelo incidente
    \item O dia e hora do incidente
\end{itemize}

No conjunto de dados inicial temos 13 colunas.



% STEP 3 IN CRISP DM METODOLOGY
\subsection{Preparação dos Dados}
No que toca à preparação de dados:
\begin{itemize}
    \item Extraímos informações extra da data/hora, com o intuito de tentar encontrar alguma associação entre o nível de incidentes e estas variáveis.
    \item Guardamos o número total de estradas afetadas, para cada incidente registado.
    \item Considerando que todos os registos são referentes á cidade de Braga não fazia sentido mantermos o campo \texttt{city\_name}, assim sendo foi removido.
    \item Atributos com pouca variância ou alta correlação foram também removidos. Tal é o caso de atributos como \texttt{avg\_atm\_pressure} ou \texttt{avg\_precipitation}.
    \item No atributo \texttt{magnitude\_of\_delay} mapeamos as classes com baixa frequência como sendo do tipo "MODERATE"
\end{itemize}

Uma vez processados os diversos atributos e aplicadas técnicas de \texttt{Feature Engeneering} ficamos com a seguinte matriz de correlação.

\begin{figure}[H]
    \centering
    \includegraphics[width=0.5\linewidth]{Figures/correlationBraga.png}
    \caption{Correlação entre features após processamento.}
    \label{fig:corr1}
\end{figure}





% STEP 4 IN CRISP DM METODOLOGY
\subsection{Modelação}
De forma a garantir a ausência de overfitting decidimos que o ideal seria considerar um modelo baseado em \textit{random forest} que, devido à sua aleatoriadade, permite reduzir fortemente o overfitting.

Com métodos auxiliares de hyper-parameter tuning, fomos capazes de descobrir os hyper-parâmetros ideias para o nosso problema e re-aplicar esses parâmetros directamente na previsão do nosso test set, obtendo uma accuracy perto de 92\%, como se verifica em capítulos adiante. 

% STEP 5 IN CRISP DM METODOLOGY
\subsection{Apreciação dos Modelos}
Consideramos que apesar da \textit{accuracy} relativamente baixa, o nosso modelo é capaz de explicar bem suficientemente a variação em vitórias, tendo em conta o pressuposto que estamos só a trabalhar com poucos minutos de jogo.