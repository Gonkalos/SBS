\section{Conclusão}
Uma vez terminada a exposição do trabalho efectuado podemos então tirar algumas conclusões sobre o mesmo. 

Como ponto inicial começamos por ressalvar a utilidade da metodologia CRISP-DM ao longo deste projeto. Serviu não só para garantir a qualidade dos modelos desenvolvidos como também para estruturar e organizar o desenvolvimento do projeto. Consideramos que esta metodologia foi, em parte, responsável pelos bons resultados obtidos, quer na exploração do \textit{dataset} relativo a acidentes em Braga em 2019, quer na exploração do \textit{dataset} relativo ao videojogo League of Legends. 

Incidindo então concretamente sobre os modelos gerados, de uma forma geral estamos satisfeitos com os resultados obtidos. Conseguimos uma precisão alta com o modelo gerado para prever a magnitude de incidentes rodoviários na cidade de Braga, obtendo uma precisão final de cerca de 92\% de acertos, reflexo também da exploração e preparação dos dados por nós feita.

Relativamente ao modelo desenhado para prever o vencedor de partidas de League of Legends, embora à primeira vista a precisão possa parecer baixa, rondando os 73\%, consideramos que foi também este um resultado bastante bom, confirmando que realmente existe um impacto grande dos 10 primeiros minutos de jogo no resultado final da partida.

Assim sendo, considerando os processos desenvolvidos, apresentados neste relatório,  e os resultados dos mesmo, consideramos que conseguimos obter resultados satisfatórios em ambos os modelos desenvolvidos. Conseguimos ainda identificar pontos que poderiam ser trabalhados/modificados de forma a conseguir resultados ainda melhores.